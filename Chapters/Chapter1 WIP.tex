% Chapter Template

\chapter{Inception} % Main chapter title

\label{Chapter1} % Change X to a consecutive number; for referencing this chapter elsewhere, use \ref{ChapterX}

%----------------------------------------------------------------------------------------

% Define some commands to keep the formatting separated from the content 
\newcommand{\keyword}[1]{\textbf{#1}}
\newcommand{\tabhead}[1]{\textbf{#1}}
\newcommand{\code}[1]{\texttt{#1}}
\newcommand{\file}[1]{\texttt{\bfseries#1}}
\newcommand{\option}[1]{\texttt{\itshape#1}}

%----------------------------------------------------------------------------------------

%----------------------------------------------------------------------------------------

% Define itemize bullet and objectives environment
\renewcommand{\labelitemi}{$\blacksquare$}
\newenvironment{myObjectives}
	{\begin{quote}
			\hrule
			\begin{center}
				\Large
				\textbf{Objectives}
			\end{center}
				\begin{itemize}
	}
	{ 			\end{itemize}
			\hrule
	\end{quote}
	}

%----------------------------------------------------------------------------------------
	
\begin{myObjectives}
	\item Establish what the software should be. Possibly what it will be in the future
	\item test
\end{myObjectives}

%----------------------------------------------------------------------------------------
%	SECTION 1
%----------------------------------------------------------------------------------------

\section{Vision}

\begin{center}
	\begin{tabularx}{\textwidth}{|c|c|>{\centering\arraybackslash}X|>{\centering\arraybackslash}X|}
		\multicolumn{4}{c}{\tabhead{Revision History}} \\
		\hline
		\tabhead{Date} &\tabhead{Version} &\tabhead{Description} &\tabhead{Author} \\
		\hline
		June 20, 2021 &Draft &Initial vision for the project. &\authorname \\
		\hline
	\end{tabularx}
\end{center} % Revision History Format

%-----------------------------------
%	SUBSECTION 1
%-----------------------------------
\subsection{Introduction}

The purpose of this document is to collect, analyze, and define high-level needs and features of the \emph{\ttitle}. It focuses on the capabilities needed by the stakeholders, and the target users, and why these needs exist. The details of how the \emph{\ttitle} fulfills these needs are detailed in the use-case and supplementary specifications.

\subsubsection{Reference}

See \artif{Glossary} % Replace with reference to glossary after glossary is made.

%-----------------------------------
%	SUBSECTION 2
%-----------------------------------

\subsection{Positioning}

\subsubsection{Problem Statement}

\begin{center}
	\begin{tabularx}{\textwidth}{|>{\hsize=.33\textwidth\raggedright\arraybackslash}X|>{\raggedleft\arraybackslash}X|}
		\hline
		\tabhead{problem of} &The lack of modernization in assessment\\
		\hline
		\tabhead{affects} &Test makers and test takers \\
		\hline
		\tabhead{the impact of which is} &That there is too much manual labor that can otherwise be automated through software. \\
		\hline
		\tabhead{a successful solution would be} &Create a software that can automate assessment, reducing time spent and improving the integrity of the results. \\
		\hline
	\end{tabularx}
\end{center}

\subsubsection{Product Position Statement}

\begin{center}
	\begin{tabularx}{\textwidth}{|>{\hsize=.33\textwidth\raggedright\arraybackslash}X|>{\raggedleft\arraybackslash}X|}
		\hline
		\tabhead{For} &Individuals or small to medium size organizations. \\
		\hline
		\tabhead{Who} &Wish to automate assessment procedures. \\
		\hline
		\tabhead{\ttitle} &Is an \emph{open-source} software product. \\
		\hline
		\tabhead{That} &That can automate assessment such as testing and evaluating. \\
		\hline
		\tabhead{Unlike} &Expensive web-base XaaS solutions which are complicated to use. Or general document editing programs which lacks essential features. \\
		\hline
		\tabhead{Our Product} &Is simple. Doesn't require programming skills to use or extensive hardware resources. Can be used \emph{off-line}. And intended for personal and small scales use. \\
		\hline
	\end{tabularx}
\end{center}

%-----------------------------------
%	SUBSECTION 3
%-----------------------------------

\subsection{Stakeholders and User Descriptions}

\subsubsection{Stake Holder Summary}

\begin{center}
	\begin{tabularx}{\textwidth}{|>{\centering\arraybackslash}X|>{\centering\arraybackslash}X|>{\centering\arraybackslash}X|}
		\hline
		\tabhead{Name} &\tabhead{Represents} &\tabhead{Role} \\
		\hline
		Developer &As of now represents the entirety of the project &Main source of funds and does everything in the project, from requirement, analysis and design, implementation, testing, deployment to project management\\
		\hline
		End-user &Represents individuals who wants to use the software, mainly me &Uses the software \\
		\hline
	\end{tabularx}
\end{center}

\subsubsection{User Summary}

\begin{center}
	\begin{tabularx}{\textwidth}{|>{\centering\arraybackslash}X|>{\centering\arraybackslash}X|>{\centering\arraybackslash}X|}
		\hline
		\tabhead{Name} &\tabhead{Description} &\tabhead{Stakeholders} \\
		\hline
		Test makers &Creates question banks and generates test Administers test. &Self-represented \\
		\hline
		Test takers &Takes the test &Self-represented \\
		\hline
		Evaluators &Evaluates the results and draws conclusions. &Self-represented \\
		\hline
	\end{tabularx}
\end{center}

\subsubsection{User Environment}

Assessment is a lengthy process and tedious at times. Often the purpose of assessment is missed altogether and \emph{passing the test} becomes the priority. This leads to assessment being reduced to a compulsory \emph{formality} instead of a tool for learning.

Users are expected to have at least a relatively \emph{low-end PC} in order to use the software. Internet connection is not required in the operations but might be needed in updating the software.

\subsubsection{Key Stakeholder or User Needs}

The \emph{needs} provided in these section is generated by myself as I am the first \emph{end-user} of the software. As it moves further in development I might elicit new needs from future stake-holders and end-users.

\begin{center}
	\begin{tabular}{|>{\bfseries\centering\arraybackslash}m{.18\textwidth}|>{\centering\arraybackslash}m{.1\textwidth}|>{\bfseries\centering\arraybackslash}m{.14\textwidth}|>{\centering\arraybackslash}m{.22\textwidth}|>{\centering\arraybackslash}m{.22\textwidth}|}
		\hline
		\tabhead{Need} &\tabhead{Priority} &\tabhead{Concerns} &\tabhead{Current Solution} &\tabhead{Proposed Solution}\\
		\hline
		Automatic test generation &High &Manual test creation is tedious and time consuming &No current solution &Employ the use a question bank and automatically generate a test questionnaire\\
		\hline
		Automatic test checking &High &Software should be able check test answers &No current solution &Answers are of course included with test item so checking the answer can be automated\\
		\hline
		Digitized test results &Low &Manual recording is time consuming and laborious &No current solution &Test results are converted to a spread sheet readable format like \brand{.csv} or can be pipe-lined to a \brand{DBMS}.
		alternatively an evaluation module can made for the software \\
		\hline
		Able to run on a low-end PC &Medium &I have a \emph{potato} laptop &No current solution &Develop for low-end PC \\
		\hline
		Usability Requirements &Low &--- &No current solution &Will be addressed in the \emph{UI Mock-ups} \\
		\hline
		Technical support needs &Low &--- &No current solution &Will be addressed during deployment stage \\
		\hline 
	\end{tabular}
\end{center}

\subsubsection{Alternatives and Competition}

The only true open-source program that can be considered a competition for this project is a \emph{computer-based testing} software called \brand{TCExam}. TCExam is a platform type software that must be run using a web-server. I wouldn't deny the fact that TCExam outclasses this project but in my opinion it fulfills a different niche than what \ttitle~ is trying to fill.

For alternatives, there are open-source documents editors such as \brand{Libre Office: Writer}, \brand{Google Docs and Google Forms}, etc $\ldots$. These software are intended for general use and lack the main features that address the needs of the end users.
%-----------------------------------
%	SUBSECTION 4
%-----------------------------------

\subsection{Product Overview}

\subsubsection{Product Perspective}

\ttitle~ is planned to be a standalone software although it would use \stndrd{Qt5 GUI API} for its \emph{UI} it's back end architecture would be designed from scratch.

\subsubsection{Assumption and Dependencies}

No dependencies to other systems as of this version.

%-----------------------------------
%	SUBSECTION 5
%-----------------------------------

\subsection{Product Features}

Will be provided at Chapter \ref{ElaborationPt.1}

%-----------------------------------
%	SUBSECTION 6
%-----------------------------------

\subsection{Other Product Requirements}

None for this version

%----------------------------------------------------------------------------------------
%	SECTION 2
%----------------------------------------------------------------------------------------

\section{Business Case}

One of the goals in the inception phase is to determine whether a new system is feasible and worth exploring. ``feasible'' is straight forward, while ``worth exploring'' is somewhat vague. To clarify the term, by saying a project is worth exploring is tantamount to saying the project is profitable. By creating a \emph{business case} both those \textbf{risk}, as they are termed in the industry, are analyzed and ultimately justified. But as it stands I find that building a \emph{business case} for this project to be redundant.

\subsection{Rationale for Leaving Out Business Case}

The rationale for leaving out \emph{business case} is that both questions asked in this section already has answers. Yes, the project is feasible, it has to be, otherwise the project would be canceled\footnote{The paper is an undergrad course requirement. The student of course has to fund it to complete the course.}. Is the project profitable? The question is invalid since the project is meant to be \emph{open-source}. Now the question is ``Is it valid to leave out the building the business case for the project?''. Yes, it is perfectly valid\footnote{biblio craid larman}. An artifact's function is to serve as utility for the project or the development of a program. In this case building a \emph{business case} doesn't provide any utility and will only consume resources.

In addition \emph{business cases} are a project by themselves. It may contain sections such as \emph{executive summary, problem statement, analysis of the situation, solution options, cost-benefit analysis, etc$\ldots$}. It is best that such studies and analysis are left to those who have expertise in the subject matter\footnote{Say students of business disciplines such as BS Accountancy or BS Business Management}. As an engineering student I think such endeavors is out of my field of study.



%----------------------------------------------------------------------------------------
%	SECTION 3
%----------------------------------------------------------------------------------------

\section{Supplementary Requirements}

Some requirement used

%----------------------------------------------------------------------------------------
%	SECTION 4
%----------------------------------------------------------------------------------------

\section{Use-case Model}

Text Model of most important use case

%----------------------------------------------------------------------------------------
%	SECTION 5
%----------------------------------------------------------------------------------------

\section{Iteration Plan}

Contains the story map from project 

 
