% Chapter Template

\chapter{Inception} % Main chapter title

\label{Chapter1} % Change X to a consecutive number; for referencing this chapter elsewhere, use \ref{ChapterX}

%----------------------------------------------------------------------------------------

% Define some commands to keep the formatting separated from the content 
\newcommand{\keyword}[1]{\textbf{#1}}
\newcommand{\tabhead}[1]{\textbf{#1}}
\newcommand{\code}[1]{\texttt{#1}}
\newcommand{\file}[1]{\texttt{\bfseries#1}}
\newcommand{\option}[1]{\texttt{\itshape#1}}

%----------------------------------------------------------------------------------------

%----------------------------------------------------------------------------------------

% Define itemize bullet and objectives environment
\renewcommand{\labelitemi}{$\blacksquare$}
\newenvironment{myObjectives}
	{\begin{quote}
			\hrule
			\begin{center}
				\Large
				\textbf{Objectives}
			\end{center}
				\begin{itemize}
	}
	{ 			\end{itemize}
			\hrule
	\end{quote}
	}

%----------------------------------------------------------------------------------------
	
\begin{myObjectives}
	\item Establish the vision, and scope of the software
	\item Provide the \artif{artifacts} created during the inception phase such as the \artif{Vision, Use Case Model, Business Case, etc$\ldots$}
	\item Justify why some artifacts have been dropped off
	\item Create a foundation for the baseline architecture
\end{myObjectives}

%----------------------------------------------------------------------------------------
%	SECTION 1
%----------------------------------------------------------------------------------------

\section{Vision}

\begin{center}
	\begin{tabularx}{\textwidth}{|c|c|>{\centering\arraybackslash}X|>{\centering\arraybackslash}X|}
		\multicolumn{4}{c}{\tabhead{Revision History}} \\
		\hline
		\tabhead{Date} &\tabhead{Version} &\tabhead{Description} &\tabhead{Author} \\
		\hline
		June 20, 2021 &Draft &Initial vision for the project. &\authorname \\
		\hline
	\end{tabularx}
\end{center} % Revision History Format

%-----------------------------------
%	SUBSECTION 1
%-----------------------------------
\subsection{Introduction}

The purpose of this document is to collect, analyze, and define high-level needs and features of the \emph{\ttitle}. It focuses on the capabilities needed by the stakeholders, and the target users, and why these needs exist. The details of how the \emph{\ttitle} fulfills these needs are detailed in the use-case and supplementary specifications.

\subsubsection{Reference}

See \artif{Glossary} \ref{Glossary} % Replace with reference to glossary after glossary is made.

%-----------------------------------
%	SUBSECTION 2
%-----------------------------------

\subsection{Positioning}

\subsubsection{Problem Statement}

\begin{center}
	\begin{tabularx}{\textwidth}{|>{\hsize=.33\textwidth\raggedright\arraybackslash}X|>{\raggedleft\arraybackslash}X|}
		\hline
		\tabhead{problem of} &The lack of modernization in assessment\\
		\hline
		\tabhead{affects} &Test makers and test takers \\
		\hline
		\tabhead{the impact of which is} &That there is too much manual labor that can otherwise be automated through software. \\
		\hline
		\tabhead{a successful solution would be} &Create a software that can automate assessment, reducing time spent and improving the integrity of the results. \\
		\hline
	\end{tabularx}
\end{center}

\subsubsection{Product Position Statement}

\begin{center}
	\begin{tabularx}{\textwidth}{|>{\hsize=.33\textwidth\raggedright\arraybackslash}X|>{\raggedleft\arraybackslash}X|}
		\hline
		\tabhead{For} &Individuals or small to medium size organizations. \\
		\hline
		\tabhead{Who} &Wish to automate assessment procedures. \\
		\hline
		\tabhead{\ttitle} &Is an \emph{open-source} software product. \\
		\hline
		\tabhead{That} &That can automate assessment such as testing and evaluating. \\
		\hline
		\tabhead{Unlike} &Expensive web-base XaaS solutions which are complicated to use. Or general document editing programs which lacks essential features. \\
		\hline
		\tabhead{Our Product} &Is simple. Doesn't require programming skills to use or extensive hardware resources. Can be used \emph{off-line}. And intended for personal and small scale uses. \\
		\hline
	\end{tabularx}
\end{center}

%-----------------------------------
%	SUBSECTION 3
%-----------------------------------

\subsection{Stakeholders and User Descriptions}

\subsubsection{Stake Holder Summary}

\begin{center}
	\begin{tabularx}{\textwidth}{|>{\centering\arraybackslash}X|>{\centering\arraybackslash}X|>{\centering\arraybackslash}X|}
		\hline
		\tabhead{Name} &\tabhead{Represents} &\tabhead{Role} \\
		\hline
		Developer &As of now represents the entirety of the project &Main source of funds and does everything in the project, from requirement, analysis and design, implementation, testing, deployment to project management\\
		\hline
		End-user &Represents individuals who wants to use the software, mainly me &Uses the software \\
		\hline
	\end{tabularx}
\end{center}

\subsubsection{User Summary}

\begin{center}
	\begin{tabularx}{\textwidth}{|>{\centering\arraybackslash}X|>{\centering\arraybackslash}X|>{\centering\arraybackslash}X|}
		\hline
		\tabhead{Name} &\tabhead{Description} &\tabhead{Stakeholders} \\
		\hline
		Test makers &Creates question banks and generates test Administers test. &Self-represented \\
		\hline
		Test takers &Takes the test &Self-represented \\
		\hline
		Evaluators &Evaluates the results and draws conclusions. &Self-represented \\
		\hline
	\end{tabularx}
\end{center}

\subsubsection{User Environment}

Assessment is a lengthy process and tedious at times. Often the purpose of assessment is missed altogether and \emph{passing the test} becomes the priority. This leads to assessment being reduced to a compulsory \emph{formality} instead of a tool for learning.

Users are expected to have at least a relatively \emph{low-end PC} in order to use the software. Internet connection is not required in the operations but might be needed in updating the software.

\subsubsection{Key Stakeholder or User Needs}

The \emph{needs} provided in these section is generated by myself as I am the first \emph{end-user} of the software. As it moves further in development I might elicit new needs from future stake-holders and end-users.

\begin{center}
	\begin{tabular}{|>{\bfseries\centering\arraybackslash}m{.18\textwidth}|>{\centering\arraybackslash}m{.1\textwidth}|>{\bfseries\centering\arraybackslash}m{.14\textwidth}|>{\centering\arraybackslash}m{.22\textwidth}|>{\centering\arraybackslash}m{.22\textwidth}|}
		\hline
		\tabhead{Need} &\tabhead{Priority} &\tabhead{Concerns} &\tabhead{Current Solution} &\tabhead{Proposed Solution}\\
		\hline
		Automatic test creation &High &Manual test creation is tedious and time consuming &No current solution &Employ the use a question bank and automatically generate a test questionnaire\\
		\hline
		Automatic test checking &High &Software should be able check test answers &No current solution &Answers are of course included with test item so checking the answer can be automated\\
		\hline
		Digitized test results &Low &Manual recording is time consuming and laborious &No current solution &Test results are converted to a spread sheet readable format like \brand{.csv} or can be pipe-lined to a \brand{DBMS}.
		alternatively an evaluation module can made for the software \\
		\hline
		User Management &Medium &User should be able to save their details along with their work &No current solution &Simple user class that can be used to save data about the user \\
		\hline
		Able to run on a low-end PC &Medium &I have a \emph{potato} laptop &No current solution &Develop for low-end PC \\
		\hline
		Usability Requirements &Low &--- &No current solution &Will be addressed in the \emph{UI Mock-ups} \\
		\hline
		Technical support needs &Low &--- &No current solution &Will be addressed during deployment stage \\
		\hline 
	\end{tabular}
\end{center}

\subsubsection{Alternatives and Competition}

The only true open-source program that can be considered a competition for this project is a \emph{computer-based testing} software called \brand{TCExam}. TCExam is a platform type software that must be run using a web-server. I wouldn't deny the fact that TCExam outclasses this project but in my opinion it fulfills a different niche than what \ttitle~ is trying to fill.

For alternatives, there are open-source documents editors such as \brand{Libre Office: Writer}, \brand{Google Docs and Google Forms}, etc $\ldots$. These software are intended for general use and lack the main features that address the needs of the end users.
%-----------------------------------
%	SUBSECTION 4
%-----------------------------------

\subsection{Product Overview}

\subsubsection{Product Perspective}

\ttitle~ is planned to be a standalone software although it would use \stndrd{Qt5 GUI API} for its \emph{UI} it's back end architecture would be designed from scratch.

\subsubsection{Assumption and Dependencies}

No dependencies to other systems as of this version.

%-----------------------------------
%	SUBSECTION 5
%-----------------------------------

\subsection{Product Features}

Will be provided at Chapter \ref{ElaborationPt.1}

%-----------------------------------
%	SUBSECTION 6
%-----------------------------------

\subsection{Other Product Requirements}

None for this version

\pagebreak
%----------------------------------------------------------------------------------------
%	SECTION 2
%----------------------------------------------------------------------------------------

\section{Business Case}

One of the goals in the inception phase is to determine whether a new system is feasible and worth exploring. ``feasible'' is straight forward, while ``worth exploring'' is somewhat vague. To clarify the term, by saying a project is worth exploring is tantamount to saying the project is profitable. By creating a \emph{business case} both those \textbf{risk}, as they are termed in the industry, are analyzed and ultimately justified. But as it stands I find that building a \emph{business case} for this project to be redundant.

\subsection{Rationale for Leaving Out Business Case}

The rationale for leaving out \emph{business case} is that both questions asked in this section already has answers. Yes, the project is feasible, it has to be, otherwise the project would be canceled\footnote{The paper is an undergrad course requirement. The student of course has to fund it to complete the course.}. Is the project profitable? The question is invalid since the project is meant to be \emph{open-source}. Now the question is ``Is it valid to leave out the building the business case for the project?''. Yes, it is perfectly valid\footnote{biblio craid larman}. An artifact's function is to serve as utility for the project or the development of a program. In this case building a \emph{business case} doesn't provide any utility and will only consume resources.

In addition \emph{business cases} are a project by themselves. It may contain sections such as \emph{executive summary, problem statement, analysis of the situation, solution options, cost-benefit analysis, etc$\ldots$}. It is best that such studies and analysis are left to those who have expertise in the subject matter\footnote{Say students of business disciplines such as BS Accountancy or BS Business Management}. As an engineering student I think such endeavors is out of my field of study.

\pagebreak

%----------------------------------------------------------------------------------------
%	SECTION 3
%----------------------------------------------------------------------------------------

\section{Use Case Model}

\begin{center}
	\begin{tabularx}{\textwidth}{|c|c|>{\centering\arraybackslash}X|>{\centering\arraybackslash}X|}
		\multicolumn{4}{c}{\tabhead{Revision History}} \\
		\hline
		\tabhead{Date} &\tabhead{Version} &\tabhead{Description} &\tabhead{Author} \\
		\hline
		June 20, 2021 &Draft &Casual form for use cases. &\authorname \\
		\hline
	\end{tabularx}
\end{center} % Revision History Format

\bigskip

\begin{center}
	\begin{flushleft}
		\Large{\textbf{UC1: Create Test}}
	\end{flushleft}
	\begin{tabularx}{\textwidth}{|>{\hspace{1em}\raggedright\arraybackslash}X}
		\hline
		\\
		\textbf{Main Success Scenario:} User starts the system. Assume that the user has already entered his details in the system. Assume also that a question bank has been provided by user. The system generates a test as per specification of the user e.g. number of sections, items per sections, type of test for each sections, etc $\ldots$. The user saves the generated test.\\
		\\
		\textbf{Alternate Scenario:} The user has not entered his details. The user then goes to \emph{subfunction} Create New User or \emph{subfunction} Change User. No question bank to generate from. The user goes to the \emph{subfunction} Create a Question Bank. Cannot generate test because question bank does not meet the specification the user has given the system. System provides hints in order to proceed with the generation. \\
	\end{tabularx}
\end{center}

\bigskip % Use Case Format

\begin{center}
	\begin{flushleft}
		\Large{\textbf{UC2: Take Test}}
	\end{flushleft}
	\begin{tabularx}{\textwidth}{|>{\hspace{1em}\raggedright\arraybackslash}X}
		\hline
		\\
		\textbf{Main Success Scenario:} User starts the system. Assume that the user has already entered his detail into the system. Assume that a generated test file is available. The system loads up the test file. The user takes the test on the systems UI or other means of output. The test result is saved along with the user details.\\
		\\
		\textbf{Alternate Scenario:} The user has not entered his details. The user then goes to \emph{subfunction} Create New User or \emph{subfunction} Change User. The user wants to check her work. \emph{Subfunction} Check Test is called.\\
	\end{tabularx}
\end{center}

\bigskip

\begin{center}
	\begin{flushleft}
		\Large{\textbf{UC3: View Scores}}
	\end{flushleft}
	\begin{tabularx}{\textwidth}{|>{\hspace{1em}\raggedright\arraybackslash}X}
		\hline
		\\
		\textbf{Main Success Scenario:} User starts the system. Assume that the user has already entered his detail in the system. The system retrieves the data requested by the user e.g. test results with the user details of test taker. The systems views the scores in a comprehensive manner e.g. graphs or reports.\\
		\\
		\textbf{Alternate Scenario:} User wants to collect the data instead of viewing it on the UI of the system or UI for viewing test result is not yet developed since it is lower in priority. \emph{Subfuntion} Other Results Output is called. \\
	\end{tabularx}
\end{center}
\bigskip

\begin{center}
	\begin{flushleft}
		\Large{\textbf{UC4: Manage Users}}
	\end{flushleft}
	\begin{tabularx}{\textwidth}{|>{\hspace{1em}\raggedright\arraybackslash}X}
		\hline
		\\
		\textbf{Main Success Scenario:} Is a collection of subfunction-level cases such as Create New User, Delete User, and Edit User Details.\\
		\\
		\textbf{Alternate Scenario:} When subfunction other than Create User is called. \\
	\end{tabularx}
\end{center}

\pagebreak
%----------------------------------------------------------------------------------------
%	SECTION 4
%----------------------------------------------------------------------------------------

\section{Supplementary Requirements}

\begin{center}
	\begin{tabularx}{\textwidth}{|c|c|>{\centering\arraybackslash}X|>{\centering\arraybackslash}X|}
		\multicolumn{4}{c}{\tabhead{Revision History}} \\
		\hline
		\tabhead{Date} &\tabhead{Version} &\tabhead{Description} &\tabhead{Author} \\
		\hline
		June 20, 2021 &Draft &Supplementary requirements during inception phase. &\authorname \\
		\hline
	\end{tabularx}
\end{center} % Revision History Format

\subsection{Introduction}

\subsubsection{Purpose}

The purpose of this document is to define requirements of the \ttitle. This Supplementary Specification lists the requirements that are not readily captured in the use cases of the use-case model. The Supplementary Specifications and the use-case model together capture a complete set of requirements on the system.

\subsection{Functionality}

Functional requirements are captured via the defined use cases.

\subsection{Usability}

\begin{description}
	\item[
Ease of Use] -- Using the software should not require any knowledge in programming. UI should be intuitive and akin to many popular software.
\end{description}

\subsection{Reliability}

\begin{description}
	\item[
Redundancy
] -- Test results and question banks should have redundancy in case of system failure to prevent lose of critical data.
\end{description}

\subsection{Performance}

\begin{description}
	\item[Low Resource Requirement] -- Software should not consume too much resources. Functionality is prioritized of over aesthetics.
\end{description}

\subsection{Supportability}

\begin{description}
	\item[Availability of Updates] -- Updates should be readily available via repository.
\end{description}

\subsection{Design Constraints}

Software must be able to run on low end PCs

\subsection{Online User Documentation and Help System Requirements}

\begin{description}
	\item[Getting Started Document
] -- At the least a ``Getting Started'' documents should be provided.
	\item[Man Pages] -- Man pages style Help for functions while UI is not yet developed.
\end{description}

\subsection{Purchased Components}

None as of this version. 

\subsection{Interface}

No UI plans as of the first iteration.

\subsection{Licensing Requirements}

Should be open source. Publicly license may be used.


\subsection{Legal Copyright and Other Notices}

None as of this version.

\subsection{Applicable Standards}

None as of this version.

\pagebreak
%----------------------------------------------------------------------------------------
%	SECTION 5
%----------------------------------------------------------------------------------------

\section{Iteration Plan}

For now I am the sole developer of the software. I find it absurd to create a schedule for each task. With regards to the timeline it also difficult to provide one with any degree of accuracy. As for milestones below is list of critical ones going into elaboration phase of \emph{iteration 1}.

\begin{itemize}
		\item Create domain model for the core architecture
		\item Convert UC1 and UC2 to fully dressed version. To define requirements even further.
		\item Implement basic key scenario mainly UC1.
		\item Implement \emph{Start Up} case for initialization needs of the system.
		\item 
\end{itemize}

 
