% Chapter Template

\chapter{Inception} % Main chapter title

\label{Chapter1} % Change X to a consecutive number; for referencing this chapter elsewhere, use \ref{ChapterX}

%----------------------------------------------------------------------------------------

% Define some commands to keep the formatting separated from the content 
\newcommand{\keyword}[1]{\textbf{#1}}
\newcommand{\tabhead}[1]{\textbf{#1}}
\newcommand{\code}[1]{\texttt{#1}}
\newcommand{\file}[1]{\texttt{\bfseries#1}}
\newcommand{\option}[1]{\texttt{\itshape#1}}

%----------------------------------------------------------------------------------------

%----------------------------------------------------------------------------------------

% Define itemize bullet and objectives environment
\renewcommand{\labelitemi}{$\blacksquare$}
\newenvironment{Objectives}
	{\begin{quote}
			\hrule
			\begin{center}
				\Large
				\textbf{Objectives}
			\end{center}
				\begin{itemize}
	}
	{ 			\end{itemize}
			\hrule
	\end{quote}
	}
%----------------------------------------------------------------------------------------
	
\begin{Objectives}
	\item Establish what the software should be. Possibly what it will be in the future
	\item test
\end{Objectives}

%----------------------------------------------------------------------------------------
%	SECTION 1
%----------------------------------------------------------------------------------------

\section{Vision}

%-----------------------------------
%	SUBSECTION 1
%-----------------------------------
\subsection{Introduction}

The purpose of this document is to collect, analyze, and define high-level needs and features of the \emph{\ttitle}. It focuses on the capabilities needed by the stakeholders, and the target users, and why these needs exist. The details of how the \emph{\ttitle} fulfills these needs are detailed in the use-case and supplementary specifications.

\subsubsection{Scope}

\subsubsection{Reference}

%-----------------------------------
%	SUBSECTION 2
%-----------------------------------

\subsection{Positioning}
Morbi rutrum odio eget arcu adipiscing sodales. Aenean et purus a est pulvinar pellentesque. Cras in elit neque, quis varius elit. Phasellus fringilla, nibh eu tempus venenatis, dolor elit posuere quam, quis adipiscing urna leo nec orci. Sed nec nulla auctor odio aliquet consequat. Ut nec nulla in ante ullamcorper aliquam at sed dolor. Phasellus fermentum magna in augue gravida cursus. Cras sed pretium lorem. Pellentesque eget ornare odio. Proin accumsan, massa viverra cursus pharetra, ipsum nisi lobortis velit, a malesuada dolor lorem eu neque.

\subsubsection{Problem Statement}

\subsubsection{Product Position Statement}

%-----------------------------------
%	SUBSECTION 3
%-----------------------------------

\subsection{Stakeholders and User Descriptions}

\subsubsection{Stake Holder Summary}

\subsubsection{User Environment}

\subsubsection{Key Stakeholder or User Needs}

\
%-----------------------------------
%	SUBSECTION 4
%-----------------------------------

\subsection{Product Overview}

%-----------------------------------
%	SUBSECTION 5
%-----------------------------------

\subsection{Product Features}

%-----------------------------------
%	SUBSECTION 6
%-----------------------------------

\subsection{Other Product Requirements}
%----------------------------------------------------------------------------------------
%	SECTION 2
%----------------------------------------------------------------------------------------

\section{Business Case}

One of the goals in the inception phase is to determine whether a new system is feasible and worth exploring. ``feasible'' is straight forward, while ``worth exploring'' is somewhat vague. To clarify the term, by saying a project is worth exploring is tantamount to saying the project is profitable. By creating a \emph{business case} both those \textbf{risk}, as they are termed in the industry, are analyzed and ultimately justified. But as it stands I find that building a \emph{business case} for this project to be redundant.

\subsection{Rationale for Leaving Out Business Case}

The rationale for leaving out \emph{business case} is that both questions asked in this section already has answers. Yes, the project is feasible, it has to be, otherwise the project would be canceled\footnote{The paper is an undergrad course requirement. The student of course has to fund it to complete the course.}. Is the project profitable? The question is invalid since the project is meant to be \emph{open-source}. Now the question is ``Is it valid to leave out the building the business case for the project?''. Yes, it is perfectly valid\footnote{biblio craid larman}. An artifact's function is to serve as utility for the project or the development of a program. In this case building a \emph{business case} doesn't provide any utility and will only consume resources.

In addition \emph{business cases} are a project by themselves. It may contain sections such as \emph{executive summary, problem statement, analysis of the situation, solution options, cost-benefit analysis, etc$\ldots$}. It is best that such studies and analysis are left to those who have expertise in the subject matter\footnote{Say students of business disciplines such as BS Accountancy or BS Business Management}. As an engineering student I think such endeavors is out of my field of study.



%----------------------------------------------------------------------------------------
%	SECTION 3
%----------------------------------------------------------------------------------------

\section{Supplementary Requirements}

Some requirement used

%----------------------------------------------------------------------------------------
%	SECTION 4
%----------------------------------------------------------------------------------------

\section{Use-case Model}

Text Model of most important use case

%----------------------------------------------------------------------------------------
%	SECTION 5
%----------------------------------------------------------------------------------------

\section{Iteration Plan}

Contains the story map from project 

 
