% Chapter Template

\chapter{Preface} % Main chapter title

\label{Preface} % Change X to a consecutive number; for referencing this chapter elsewhere, use \ref{ChapterX}

%----------------------------------------------------------------------------------------

% Define some commands to keep the formatting separated from the content 
\newcommand{\keyword}[1]{\textbf{#1}}
\newcommand{\tabhead}[1]{\textbf{#1}}
\newcommand{\code}[1]{\texttt{#1}}
\newcommand{\file}[1]{\texttt{\bfseries#1}}
\newcommand{\option}[1]{\texttt{\itshape#1}}

%----------------------------------------------------------------------------------------


The Preface is an optional read but contains vital information on how the paper was structured, the resources used in creating the software and writing th paper itself, and finally additional resources such as the online repository of the project.

%----------------------------------------------------------------------------------------
%	SECTION 1
%----------------------------------------------------------------------------------------
\section*{Structure of the Paper}
Although the paper is labeled as thesis it is virtually a capstone project. This is more evident when the only requirement given is that it should have a prototype. And so the suggested format from the article ``Guidelines in Writing Design Project / Research Project Proposal'' doesn't seem to frame the project  properly. Be that as it may, in order to comply and frame the project properly, I superimposed the \emph{Universal Process Phases:\bf{Inception, Elaboration, Construction, and Transition}} over the suggested format in the following manner:
\begin{description}
	\item[Chapter I--The Problem and Its Setting \emph{superimposed with} Inception] -- Contains the subsections \emph{Vision, Business Case, Supplementary Requirements, Use-case Model,} and \emph{Iteration Plan} which replaced the subsections \emph{Introduction, Statement of the Problem, Objectives of the Study, Importance of the Study,} and \emph{Scope and Limitations of the Study} respectively. The lack of the subsections \emph{Definition of Terms} and \emph{Conceptual Framework} is addressed on the next chapter.
	\item[Chapter II--The Problem and Its Setting \emph{superimposed with} Elaboration Pt. 1] -- This chapter completely deviates from the suggested format since there are no hypothesis to prove or a conclusion to draw. There are also no conceptual framework on which the project can be framed into. Instead this chapter is spent on creating that \emph{conceptual domain} with the help of a \emph{class diagram} in \textbf{UML}.
	\item[Chapter III--Methodology \emph{superimposed with} Elaboration Pt.2 and Construction] -- description
\end{description}

In addition the manual of the software can be found in ``Cross Ref'' appendix

%-----------------------------------
%	SUBSECTION 1
%-----------------------------------
%\subsection{Subsection 1}

%Nunc posuere quam at lectus tristique eu ultrices augue venenatis. Vestibulum ante ipsum primis in faucibus orci luctus et ultrices posuere cubilia Curae; Aliquam erat volutpat. Vivamus sodales tortor eget quam adipiscing in vulputate ante ullamcorper. Sed eros ante, lacinia et sollicitudin et, aliquam sit amet augue. In hac habitasse platea dictumst.

%-----------------------------------
%	SUBSECTION 2
%-----------------------------------

%\subsection{Subsection 2}
%Morbi rutrum odio eget arcu adipiscing sodales. Aenean et purus a est pulvinar pellentesque. Cras in elit neque, quis varius elit. Phasellus fringilla, nibh eu tempus venenatis, dolor elit posuere quam, quis adipiscing urna leo nec orci. Sed nec nulla auctor odio aliquet consequat. Ut nec nulla in ante ullamcorper aliquam at sed dolor. Phasellus fermentum magna in augue gravida cursus. Cras sed pretium lorem. Pellentesque eget ornare odio. Proin accumsan, massa viverra cursus pharetra, ipsum nisi lobortis velit, a malesuada dolor lorem eu neque.

%----------------------------------------------------------------------------------------
%	SECTION 2
%----------------------------------------------------------------------------------------

\section*{Software, Standards, and Technology Used }
Below is a table enumerating all the software, standards or languages, and other technologies used in the creating the software and the paper itself.

\begin{table}
	\caption{Resources Used in the Project}
	\label{tab:Resources}
	\centering
	\begin{tabular}{l l l}
		\toprule
		\tabhead{Resource} & \tabhead{Function} & \tabhead{Link} \\
		\midrule
		Qt5 & UI IDE & some link\\
		\bottomrule\\
	\end{tabular}
\end{table}


%----------------------------------------------------------------------------------------
%	SECTION 3
%----------------------------------------------------------------------------------------
\section*{Repository}
Repo Link
\end{abstract}

